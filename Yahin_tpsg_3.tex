\documentclass[bachelor, och, coursework]{shiza}
% параметр - тип обучения - одно из значений:
%    spec     - специальность
%    bachelor - бакалавриат (по умолчанию)
%    master   - магистратура
% параметр - форма обучения - одно из значений:
%    och   - очное (по умолчанию)
%    zaoch - заочное
% параметр - тип работы - одно из значений:
%    referat    - реферат
%    coursework - курсовая работа (по умолчанию)
%    diploma    - дипломная работа
%    pract      - отчет по практике
% параметр - включение шрифта
%    times    - включение шрифта Times New Roman (если установлен)
%               по умолчанию выключен
\usepackage{subfigure}
\usepackage{tikz,pgfplots}
\pgfplotsset{compat=1.5}
\usepackage{float}
\usepackage{pdfpages}

%\usepackage{titlesec}
\setcounter{secnumdepth}{4}
%\titleformat{\paragraph}
%{\normalfont\normalsize}{\theparagraph}{1em}{}
%\titlespacing*{\paragraph}
%{35.5pt}{3.25ex plus 1ex minus .2ex}{1.5ex plus .2ex}

\titleformat{\paragraph}[block]
{\hspace{1.25cm}\normalfont}
{\theparagraph}{1ex}{}
\titlespacing{\paragraph}
{0cm}{2ex plus 1ex minus .2ex}{.4ex plus.2ex}

% --------------------------------------------------------------------------%


\usepackage[T2A]{fontenc}
\usepackage[utf8]{inputenc}
\usepackage{graphicx}
\graphicspath{ {./images/} }
\usepackage{tempora}

\usepackage[sort,compress]{cite}
\usepackage{amsmath}
\usepackage{amssymb}
\usepackage{amsthm}
\usepackage{fancyvrb}
\usepackage{listings}
\usepackage{listingsutf8}
\usepackage{longtable}
\usepackage{array}
\usepackage[english,russian]{babel}

%\usepackage[colorlinks=true]{hyperref}
\usepackage{url}

\usepackage{underscore}
\usepackage{setspace}
\usepackage{indentfirst} 
\usepackage{mathtools}
\usepackage{amsfonts}
\usepackage{enumitem}
\usepackage{tikz}
\usepackage{multirow}
\usepackage{colortbl}

\usepackage{minted}
\setminted{style=vs,autogobble=true,fontsize=\footnotesize,breaklines=true}
\newcommand{\eqdef}{\stackrel {\rm def}{=}}
\newcommand{\specialcell}[2][c]{%
	\begin{tabular}[#1]{@{}c@{}}#2\end{tabular}}

\renewcommand\theFancyVerbLine{\small\arabic{FancyVerbLine}}

\newtheorem{lem}{Лемма}

\lstset{%
	language=C,
	basicstyle=\ttfamily\fontsize{11pt}{11pt}\selectfont,
	%basicstyle=\footnotesize\ttfamily,
	breaklines=true,
	%backgroundcolor=\color{gray!12},
	basicstyle=\ttfamily\small,
	keywordstyle=\color{blue},
	commentstyle= \color{green},
	stringstyle=\color{orange},
	%identifierstyle=\color{yellow},
	showstringspaces=false,
	captionpos=b,
	%numbers=left,
	%numberstyle=\footnotesize\color{gray},
	%frame=TB,
	tabsize=2,
	%keepspaces=true,
	morekeywords={procedure, then, begin, end}
}


\begin{document}
	
	% Кафедра (в родительном падеже)
	%\chair{теоритических основ компьютерной безопасности и криптографии}
	
	% Тема работы
	%\title{Разделение секрета с помощью схемы Шамира}
	
	% Курс
	%\course{3}
	
	% Группа
	%\group{331}
	
	% Факультет (в родительном падеже) (по умолчанию "факультета КНиИТ")
	%\department{факультета КНиИТ}
	
	% Специальность/направление код - наименование
	%\napravlenie{09.03.04 "--- Программная инженерия}
	%\napravlenie{010500 "--- Математическое обеспечение и администрирование информационных систем}
	%\napravlenie{230100 "--- Информатика и вычислительная техника}
	%\napravlenie{231000 "--- Программная инженерия}
	%\napravlenie{100501 "--- Компьютерная безопасность}
	
	% Для студентки. Для работы студента следующая команда не нужна.
	% \studenttitle{Студентки}
	
	% Фамилия, имя, отчество в родительном падеже
	%\author{Яхина Шамиля Илдусовича}
	
	% Заведующий кафедрой
	%\chtitle{доцент, к.ф.-м.н.} % степень, звание
	%\chname{М.~Б.~Абросимов}
	
	%Научный руководитель (для реферата преподаватель проверяющий работу)
	%\satitle{ассистент} %должность, степень, звание
	%\saname{А. А. Лобов}
	
	% Руководитель практики от организации (только для практики,
	% для остальных типов работ не используется)
	%\patitle{к.ф.-м.н.}
	%\paname{М.~Б.~Абросимов}
	
	% Семестр (только для практики, для остальных
	% типов работ не используется)
	%\term{8}
	
	% Наименование практики (только для практики, для остальных
	% типов работ не используется)
	%\practtype{преддипломная}
	
	% Продолжительность практики (количество недель) (только для практики,
	% для остальных типов работ не используется)
	%\duration{4}
	
	% Даты начала и окончания практики (только для практики, для остальных
	% типов работ не используется)
	%\practStart{30.04.2019}
	%\practFinish{27.05.2019}
	
	% Год выполнения отчета
	%\date{2022}
	
	%\maketitle
	
	% Включение нумерации рисунков, формул и таблиц по разделам
	% (по умолчанию - нумерация сквозная)
	% (допускается оба вида нумерации)
	% \secNumbering
	\includepdf{title-3.pdf}
	
	%-------------------------------------------------------------------------------------------
	%\tableofcontents
	%\includepdf{table-of-content.pdf}
	%ПЛАН ДЛЯ КУРСОВОЙ:
		
	\section{Тестирование статистических свойств последовательности псевдослучайных чисел}
	
	Описание задания:
	
	Протестировать статистические свойства последовательности псевдослучайных чисел:
	\begin{enumerate}
		\item Вычислить математическое ожидание последовательности;
		\item Вычислить среднеквадратичное отклонение последовательности;
		\item Сравните полученные оценки с заданными в пп. 1 параметрами. Постройте графики зависимостей
		оценок от объема выборки. Оцените относительные погрешности для какой-либо одной выборки.
		\item Вычислить значение и дать ответ на вопрос удовлетворяет ли ППСЧ
		\begin{enumerate}
			\item Критерию хи-квадрат;
			\item Критерию серий;
			\item Критерию интервалов;
			\item Критерию разбиений;
			\item Критерию перестановок;
			\item Критерию монотонности;
			\item Критерию конфликтов.
		\end{enumerate}
	\end{enumerate} \

	Описание используемых ПСЧ:
	
	В работе рассматриваются последовательности псевдослучайных величин, сгенерированные в первой части практической работы следующими алгоритмами:
	\begin{enumerate}
		\item Линейный конгруэнтный метод;
		\item Аддитивный метод;
		\item Пятипараметрический метод;
		\item Регистр сдвига с обратной связью (РСЛОС);
		\item Нелинейная комбинация РСЛОС;
		\item Вихрь Мерсенна;
		\item RC4;
		\item ГПСЧ на основе RSA;
		\item Алгоритм Блюма-Блюма-Шуба.
	\end{enumerate}

	После генерации данные последовательности с помощью программы из второй части практической работы были приведены к стандартному равномерному распределению с входными параметрами 0 и 0,999, т.е. каждое число $c$ из последовательности $\in [0, 0.999]$.
	
	\subsection{Мат. ожидание, среднекв. отклонение, сравнение с теоретическими оценками и построение графиков}
	
	Для того, чтобы найти \textbf{математическое ожидание} необходимо воспользоваться следующей формулой:
	
	\begin{center}
		$M = \dfrac{\sum_{i = 1}^{k} x_i}{k}$,
	\end{center}

	где $x_i$ - числа последовательности, $k$ - количество таких чисел.
	
	Для того, чтобы найти \textbf{среднеквадратичное отклонение} необходимо воспользоваться следующей формулой:
	
	\begin{center}
		$\sigma = \sqrt{\dfrac{\sum_{i = 1}^{k} |x_i - M|^2}{k}}$,
	\end{center}
	
	где $x_i$ - числа последовательности, $k$ - количество таких чисел, $M$ - математическое ожидание.
	
	Теоретическая оценка при равномерном распределении для мат. ожидания равна 0.5, для среднеквадратичного отлонения примерно равна 0.2887.
	
	Далее необходимо посчитать относительную погрешность для одной выборки. Для этого необходимо сравнить полученные результаты с эталонными и вычислить погрешности согласно следующим формулам: 
	
		\begin{center}
			$\Delta M = \dfrac{|M - 0.5|}{M}$, $\Delta Q = \dfrac{|Q - \sqrt{\dfrac{1}{12}}|}{M}$,
		\end{center}
			
	 где $M$ – мат. ожидание, а $Q$ – среднекв. отклонение.
	
	Посчитаем данные показатели для рассматриваемых последовательностей:

	\begin{center}
		\begin{tabular}{|c|c|c|c|c|}
			\hline
			Генератор & \shortstack{ Мат. \\ ожидание} & \shortstack{Среднекв. \\ отклонение} & \multicolumn{2}{c|}{Погрешность}\\
			\cline{4-5}
			& & & \shortstack{Мат. \\ ожидания} & \shortstack{Среднекв. \\ отклонения}\\
			\hline
			lc & 0.497 & 0.287 & 0.006 & 0.005 \\
			\hline
			add & 0.496 & 0.287 & 0.009 & 0.007 \\
			\hline
			5p & 0.499 & 0.289 & 0.002 & 0.001 \\
			\hline
			lfsr & 0.496 & 0.286 & 0.008 & 0.008 \\
			\hline
			nfsr & 0.503 & 0.288 & 0.007 & 0.002 \\
			\hline
			mt & 0.501 & 0.287 & 0.002 & 0.005 \\
			\hline
			rc4 & 0.508 & 0.284 & 0.015 & 0.017 \\
			\hline
			rsa & 0.508 & 0.288 & 0.015 & 0.002 \\
			\hline
			bbs & 0.505 & 0.289 & 0.009 & 0.002 \\
			\hline
		\end{tabular}
	\end{center}
		
	\begin{figure}[H]
		\centering
		\includegraphics[width=0.9\textwidth]{MatLc}
		\caption{Вычисление показателей для генератора lc}
		\label{fig:MatLc}
	\end{figure}

	\begin{figure}[H]
		\centering
		\includegraphics[width=0.9\textwidth]{MatAdd}
		\caption{Вычисление показателей для генератора add}
		\label{fig:MatAdd}
	\end{figure}
	
	\begin{figure}[H]
		\centering
		\includegraphics[width=0.9\textwidth]{Mat5p}
		\caption{Вычисление показателей для генератора 5p}
		\label{fig:Mat5p}
	\end{figure}	
	
	\begin{figure}[H]
		\centering
		\includegraphics[width=0.9\textwidth]{MatLfsr}
		\caption{Вычисление показателей для генератора lfsr}
		\label{fig:MatLfsr}
	\end{figure}
	
	\begin{figure}[H]
		\centering
		\includegraphics[width=0.9\textwidth]{MatNfsr}
		\caption{Вычисление показателей для генератора nfsr}
		\label{fig:MatNfsr}
	\end{figure}
	
	\begin{figure}[H]
		\centering
		\includegraphics[width=0.9\textwidth]{MatMt}
		\caption{Вычисление показателей для генератора mt}
		\label{fig:MatMt}
	\end{figure}		
	
	\begin{figure}[H]
		\centering
		\includegraphics[width=0.9\textwidth]{MatRc4}
		\caption{Вычисление показателей для генератора rc4}
		\label{fig:MatRc4}
	\end{figure}
	
	\begin{figure}[H]
		\centering
		\includegraphics[width=0.9\textwidth]{MatRsa}
		\caption{Вычисление показателей для генератора rsa}
		\label{fig:MatRsa}
	\end{figure}
	
	\begin{figure}[H]
		\centering
		\includegraphics[width=0.9\textwidth]{MatBbs}
		\caption{Вычисление показателей для генератора bbs}
		\label{fig:MatBbs}
	\end{figure}	

	Далее построим графики зависимости мат. ожидания от объема выборки и среднеквадратичного отклонения от объема выборки. Шаг: 50.	
	
	\begin{figure}[H]
		\centering
		\includegraphics[width=0.9\textwidth]{GrLc}
		\caption{Графики зависимости для генератора lc}
		\label{fig:GrLc}
	\end{figure}
	
	\begin{figure}[H]
		\centering
		\includegraphics[width=0.9\textwidth]{GrAdd}
		\caption{Графики зависимости для генератора add}
		\label{fig:GrAdd}
	\end{figure}
	
	\begin{figure}[H]
		\centering
		\includegraphics[width=0.9\textwidth]{Gr5p}
		\caption{Графики зависимости для генератора 5p}
		\label{fig:Gr5p}
	\end{figure}	
	
	\begin{figure}[H]
		\centering
		\includegraphics[width=0.9\textwidth]{GrLfsr}
		\caption{Графики зависимости для генератора lfsr}
		\label{fig:GrLfsr}
	\end{figure}
	
	\begin{figure}[H]
		\centering
		\includegraphics[width=0.9\textwidth]{GrNfsr}
		\caption{Графики зависимости для генератора nfsr}
		\label{fig:GrNfsr}
	\end{figure}
	
	\begin{figure}[H]
		\centering
		\includegraphics[width=0.9\textwidth]{GrMt}
		\caption{Графики зависимости для генератора mt}
		\label{fig:GrMt}
	\end{figure}		
	
	\begin{figure}[H]
		\centering
		\includegraphics[width=0.9\textwidth]{GrRc4}
		\caption{Графики зависимости для генератора rc4}
		\label{fig:GrRc4}
	\end{figure}
	
	\begin{figure}[H]
		\centering
		\includegraphics[width=0.9\textwidth]{GrRsa}
		\caption{Графики зависимости для генератора rsa}
		\label{fig:GrRsa}
	\end{figure}
	
	\begin{figure}[H]
		\centering
		\includegraphics[width=0.9\textwidth]{GrBbs}
		\caption{Графики зависимости для генератора bbs}
		\label{fig:GrBbs}
	\end{figure}	
	
	\subsection{Сравнение полученных оценок с теоретическими и построение графиков}
	
	\subsection{Проверка критериев}
	
	\subsubsection{Критерий хи-квадрат}
	
	Описание критерия:
	
	Проверка критерия $\chi^2$ для некоторой последовательности чисел (или наблюдений величины Х) будет состоять из следующих шагов:
	
	\begin{enumerate}
		\item Выполняем достаточное число независимых наблюдений.
		\item Подсчитываем число $n_i$ наблюдений попавших в каждый из интервалов $(a_i, b_i]$, $i = 1, \dots , k$.
		\item Подсчитываем статистику
				\begin{center}
					$\chi^2 = \sum_{j = 1}^{k} \dfrac{(n_j - E_j)^2}{E_j}$;
				\end{center}
		\item Определяем, находится ли вычисленная в доверительном интервале.
	\end{enumerate} \

	Приведем результаты выполнения программы для каждого из генераторов:
	
	\begin{figure}[H]
		\centering
		\includegraphics[width=0.9\textwidth]{ChiLc}
		\caption{Результаты выполнения программы для генератора lc}
		\label{fig:ChiLc}
	\end{figure}
	
	\begin{figure}[H]
		\centering
		\includegraphics[width=0.9\textwidth]{ChiAdd}
		\caption{Результаты выполнения программы для генератора add}
		\label{fig:ChiAdd}
	\end{figure}
	
	\begin{figure}[H]
		\centering
		\includegraphics[width=0.9\textwidth]{Chi5p}
		\caption{Результаты выполнения программы для генератора 5p}
		\label{fig:Chi5p}
	\end{figure}	
	
	\begin{figure}[H]
		\centering
		\includegraphics[width=0.9\textwidth]{ChiLfsr}
		\caption{Результаты выполнения программы для генератора lfsr}
		\label{fig:ChiLfsr}
	\end{figure}
	
	\begin{figure}[H]
		\centering
		\includegraphics[width=0.9\textwidth]{ChiNfsr}
		\caption{Результаты выполнения программы для генератора nfsr}
		\label{fig:ChiNfsr}
	\end{figure}
	
	\begin{figure}[H]
		\centering
		\includegraphics[width=0.9\textwidth]{ChiMt}
		\caption{Результаты выполнения программы для генератора mt}
		\label{fig:ChiMt}
	\end{figure}		
	
	\begin{figure}[H]
		\centering
		\includegraphics[width=0.9\textwidth]{ChiRc4}
		\caption{Результаты выполнения программы для генератора rc4}
		\label{fig:ChiRc4}
	\end{figure}
	
	\begin{figure}[H]
		\centering
		\includegraphics[width=0.9\textwidth]{ChiRsa}
		\caption{Результаты выполнения программы для генератора rsa}
		\label{fig:ChiRsa}
	\end{figure}
	
	\begin{figure}[H]
		\centering
		\includegraphics[width=0.9\textwidth]{ChiBbs}
		\caption{Результаты выполнения программы для генератора bbs}
		\label{fig:ChiBbs}
	\end{figure}
	
	Исходный код программы: \
	

	\begin{minted}[fontsize=\footnotesize]{cpp}
		bool checkXiSquareCriterion(vector <double> allElems) {
			double maxElem = -1.0, minElem = 1.0;
			int elemsCount = allElems.size();
			for (int i = 0; i < elemsCount; i++)
			{
				double checkElem = allElems[i];
				if (checkElem > maxElem)
				maxElem = checkElem;
				if (checkElem < minElem)
				minElem = checkElem;
			}
			vector <pair <double, double>> allIntervals;
			int intervalsCount = ceil(1 + 1.4 * log(elemsCount));
			double IntS = (round(((maxElem - minElem) / intervalsCount) * 10000)) / 10000;
			for (int i = 0; i < intervalsCount; i++) {
				double a_i = minElem + IntS * i;
				double b_i = minElem + IntS * (i + 1);
				allIntervals.push_back(make_pair(a_i, b_i));
			}
			double expectedNumOfHits = round((elemsCount / intervalsCount) * 1000) / 1000;
			vector <int> realDistribution(intervalsCount, 0);
			for (int i = 0; i < elemsCount; i++)
			for (int j = 0; j < intervalsCount; j++)
			if (allIntervals[j].first <= allElems[i] && allElems[i] < allIntervals[j].second) {
				realDistribution[j]++;
				break;
			}
			double XiSquareDistribution = 0;
			for (int i = 0; i < intervalsCount; i++)
			XiSquareDistribution = XiSquareDistribution + ((realDistribution[i] - expectedNumOfHits) * (realDistribution[i] - expectedNumOfHits)) / expectedNumOfHits;
			XiSquareDistribution = round(XiSquareDistribution * 10000) / 10000;
			double criticalImportance = 22.36203;
			cout << "\n > Количество интервалов: " << intervalsCount;
			int k = intervalsCount - 1;
			cout << "\n > Ожидаемое число попаданий в интервалы: " << expectedNumOfHits;
			cout << "\n > Число попаданий в каждый из интервалов: \n";
			for (int i = 0; i < intervalsCount; i++)
			cout << "    " << i + 1 << ": " << realDistribution[i] << endl;
			cout << "\n > Критическое значение хи-квадрат для " << k << " степеней свобод: " << criticalImportance;
			cout << "\n > Значение критерия хи-квадрат с " << k << " степенью свободы: " << XiSquareDistribution << "\n";
			if (0 < XiSquareDistribution && XiSquareDistribution < criticalImportance) {
				cout << "\n[+++] Критерий пройден\n";
				return true;
			}
			else {
				cout << "\n[---] Критерий не пройден\n";
				return false;
			}
		}
	\end{minted}	
	
	\subsubsection{Критерий серий}	
	
	Описание критерия:

	Критерий серий позволяет убедиться в том, что пары последовательных чисел равномерно
	распределены независимым образом. Проверка критерия проводится следующим образом: 
	
	\begin{enumerate}
		\item Воспользуемся критерием Хи-квадрат. Для этого преобразуем последовательность $X^n = (x_1, x_2, \dots, x_n)$ в $Y^n = (d[x_1], d[x_2], \dots, d[n])$ с некоторым $d$ (в реализации $d$ = 4).
		\item Подсчитываем количество совпадений 
				\begin{center}
					$(y_{2j}, y_{2j+1}) = (q, r),	0 \leq j \leq n, 0 \leq q, r \leq d$.
				\end{center}
		\item Применяем Хи-квадрат критерий к полученному набору с параметрами:
				\begin{center}
					$k = d^2$, $p_j = \dfrac{1}{d^2}$;
				\end{center}
	\end{enumerate}
	
	Приведем результаты выполнения программы для каждого из генераторов:
	
	\begin{figure}[H]
		\centering
		\includegraphics[width=0.9\textwidth]{SerLc}
		\caption{Результаты выполнения программы для генератора lc}
		\label{fig:SerLc}
	\end{figure}
	
	\begin{figure}[H]
		\centering
		\includegraphics[width=0.9\textwidth]{SerAdd}
		\caption{Результаты выполнения программы для генератора add}
		\label{fig:SerAdd}
	\end{figure}
	
	\begin{figure}[H]
		\centering
		\includegraphics[width=0.9\textwidth]{Ser5p}
		\caption{Результаты выполнения программы для генератора 5p}
		\label{fig:Ser5p}
	\end{figure}	
	
	\begin{figure}[H]
		\centering
		\includegraphics[width=0.9\textwidth]{SerLfsr}
		\caption{Результаты выполнения программы для генератора lfsr}
		\label{fig:SerLfsr}
	\end{figure}
	
	\begin{figure}[H]
		\centering
		\includegraphics[width=0.9\textwidth]{SerNfsr}
		\caption{Результаты выполнения программы для генератора nfsr}
		\label{fig:SerNfsr}
	\end{figure}
	
	\begin{figure}[H]
		\centering
		\includegraphics[width=0.9\textwidth]{SerMt}
		\caption{Результаты выполнения программы для генератора mt}
		\label{fig:SerMt}
	\end{figure}		
	
	\begin{figure}[H]
		\centering
		\includegraphics[width=0.9\textwidth]{SerRc4}
		\caption{Результаты выполнения программы для генератора rc4}
		\label{fig:SerRc4}
	\end{figure}
	
	\begin{figure}[H]
		\centering
		\includegraphics[width=0.9\textwidth]{SerRsa}
		\caption{Результаты выполнения программы для генератора rsa}
		\label{fig:SerRsa}
	\end{figure}
	
	\begin{figure}[H]
		\centering
		\includegraphics[width=0.9\textwidth]{SerBbs}
		\caption{Результаты выполнения программы для генератора bbs}
		\label{fig:SerBbs}
	\end{figure}
	
	Исходный код программы:
	
	\begin{minted}[fontsize=\footnotesize]{cpp}
		bool checkSeriesCriterion(vector <double> allElems) {
			int d = 4;
			vector <vector <int>> nPairs(d);
			for (int i = 0; i < d; i++)
			nPairs[i].resize(d, 0);
			int k = d * d;
			vector <pair <int, int>> pairCategories;
			int elemsCount = allElems.size();
			if (elemsCount % 2 != 0)
			elemsCount--;
			for (int j = 0; j < elemsCount; j+=2)
			{
				double leftI = floor(allElems[j] * d);
				double rightI = floor(allElems[j + 1] * d);
				pairCategories.push_back(make_pair(leftI, rightI));
			}
			double criticalImportance = 26.29623;
			double expectedNumOfHits = round((elemsCount / (2 * k)) * 1000) / 1000;
			for (int i = 0; i < pairCategories.size(); i++) {
				nPairs[pairCategories[i].first][pairCategories[i].second] += 1;
			}
			double XiSquareDistribution = 0;
			for (int i = 0; i < nPairs.size(); i++)
			for (int x = 0; x < nPairs[i].size(); x++)
			XiSquareDistribution += ((nPairs[i][x] - expectedNumOfHits) * (nPairs[i][x] - expectedNumOfHits)) / expectedNumOfHits;
			XiSquareDistribution = round(XiSquareDistribution * 1000) / 1000;
			cout << "\n > Параметр d: " << d;
			cout << "\n > Ожидаемое количество чисел в каждой категории: " << expectedNumOfHits;
			cout << "\n > Количество чисел в каждой категории: \n";
			int chCount = 0;
			for (int i = 0; i < nPairs.size(); i++) {
				for (int x = 0; x < nPairs[i].size(); x++) {
					chCount++;
					cout << "   " << chCount << ": " << nPairs[i][x] << "\n";
				}
				chCount++;
			}
			cout << "\n > Критическое значение хи-квадрат для " << k << " степеней свобод: " << criticalImportance;
			cout << "\n > Значение критерия хи-квадрат с " << k << " степенью свободы: " << XiSquareDistribution << "\n";
			if (0 < XiSquareDistribution && XiSquareDistribution < criticalImportance) {
				cout << "\n[+++] Критерий пройден\n";
				return true;
			}
			else {
				cout << "\n[---] Критерий не пройден\n";
				return false;
			}
		}
	\end{minted}
	
	\subsubsection{Критерий интервалов}
		
	Описание критерия:
	
	Пусть $a$ и $b$ – два действительных числа таких, что $0 \leq a < b \leq 1$. Рассмотрим длины подпоследовательностей $x_j, x_{j + 1}, \dots, x_{j + r}$, в которых $x_j, x_{j + 1}, \dots,$ $x_{j + r - 1} \notin [a, b], x_{j + r} \in [a, b]$. Такую последовательность будем назы-
	вать интервалом длины $r$.
	
	Сначала, нам нужно подсчитать число интервалов длиной $0, 1, \dots, n$.
	
	Шаги алгоритма подсчета числа интервалов:
	\begin{enumerate}
		\item Инициализация. Присвоить $j$ = -1, $s$ = 0, $c_r$ = 0, $0 \leq r \leq t$.
		\item $r$ = 0.
		\item $j$ = $j$ + 1. Если $a \leq x_j \leq b$, то переход на шаг 5.
		\item $r$ = $r$ + 1. Переход к шагу 3.
		\item Если $r \leq t$, то $c_t$ = $c_t$ + 1, иначе – $c_r$ = $c_r$ + 1.
		\item $s$ = $s$ + 1. Если $s < n$  то переход на шаг 2.
	\end{enumerate}

	После этого мы можем применить хи-квадрат критерий для $k$ = $t$ + 1 к
	значениям $c_i$, $i$ = $0, 1, \dots , t$ с параметрами
			\begin{center}
				$p_r = p (1 - p) r$ для $0 \leq r \leq t - 1$;
				$p_t = (1 - p) t$;
				$p = (a - b)$.
			\end{center}	

	Здесь $p$ – вероятность того, что $a \leq x_j \leq b$. Значения $n$ и $t$ выбираются
	так, чтобы ожидаемое значение $c_r$ было больше 5. \
	
	Приведем результаты выполнения программы для каждого из генераторов:
	
	\begin{figure}[H]
		\centering
		\includegraphics[width=0.9\textwidth]{IntLc}
		\caption{Результаты выполнения программы для генератора lc}
		\label{fig:IntLc}
	\end{figure}
	
	\begin{figure}[H]
		\centering
		\includegraphics[width=0.9\textwidth]{IntAdd}
		\caption{Результаты выполнения программы для генератора add}
		\label{fig:IntAdd}
	\end{figure}
	
	\begin{figure}[H]
		\centering
		\includegraphics[width=0.9\textwidth]{Int5p}
		\caption{Результаты выполнения программы для генератора 5p}
		\label{fig:Int5p}
	\end{figure}	
	
	\begin{figure}[H]
		\centering
		\includegraphics[width=0.9\textwidth]{IntLfsr}
		\caption{Результаты выполнения программы для генератора lfsr}
		\label{fig:IntLfsr}
	\end{figure}
	
	\begin{figure}[H]
		\centering
		\includegraphics[width=0.9\textwidth]{IntNfsr}
		\caption{Результаты выполнения программы для генератора nfsr}
		\label{fig:IntNfsr}
	\end{figure}
	
	\begin{figure}[H]
		\centering
		\includegraphics[width=0.9\textwidth]{IntMt}
		\caption{Результаты выполнения программы для генератора mt}
		\label{fig:IntMt}
	\end{figure}		
	
	\begin{figure}[H]
		\centering
		\includegraphics[width=0.9\textwidth]{IntRc4}
		\caption{Результаты выполнения программы для генератора rc4}
		\label{fig:IntRc4}
	\end{figure}
	
	\begin{figure}[H]
		\centering
		\includegraphics[width=0.9\textwidth]{IntRsa}
		\caption{Результаты выполнения программы для генератора rsa}
		\label{fig:IntRsa}
	\end{figure}
	
	\begin{figure}[H]
		\centering
		\includegraphics[width=0.9\textwidth]{IntBbs}
		\caption{Результаты выполнения программы для генератора bbs}
		\label{fig:IntBbs}
	\end{figure}
	
	Исходный код программы:
	
	\begin{minted}[fontsize=\footnotesize]{cpp}
		bool checkIntervalsCriterion(vector <double> allElems) {
			int t = 10, n = 1000;
			srand((unsigned int)time(0));
			double alpha = (double)(rand()) / RAND_MAX, beta = (double)(rand()) / RAND_MAX;
			while (0.2 >= abs(alpha - beta) || abs(alpha - beta) >= 0.4) {
				alpha = (double)(rand()) / RAND_MAX;
				beta = (double)(rand()) / RAND_MAX;
			}
			if (beta < alpha) {
				double dopD = alpha;
				alpha = beta;
				beta = dopD;
			}
			alpha = round(alpha * 1000) / 1000;
			beta = round(beta * 1000) / 1000;
			double p = beta - alpha;
			int s = 0;
			vector <int> count, resCount;
			double XiSquareDistribution = 100000000000.0;
			double criticalImportance = 100000000000.0;
			int resT = 100000000000;
			int intervalsCount = 100000000000;
			vector <double> resPs;
			s = 0;
			count.resize(t + 1, 0);
			int r = 0;
			for (int j = 0; j < allElems.size(); j++) {
				if (alpha <= allElems[j] && allElems[j] < beta) {
					if (r >= t)
					count[t]++;
					else
					count[r]++;
					s++;
					if (s < n)
					r = 0;
					else
					break;
				}
				else
				r++;
			}
			vector <double> probs;
			for (int j = 0; j < t; j++)
			probs.push_back(round((p * (powDouble(1.0 - p, j))) * 1000) / 1000);
			double XiSch = 0.0;
			probs.push_back(powDouble(1.0 - p, t));
			double criticalImpDop = InterVector[t];
			int qN = 0;
			for (int j = 0; j < count.size(); j++)
			qN += count[j];
			for (int j = 0; j < probs.size(); j++) {
				double est = qN * probs[j];
				if (est == 0.0)
				break;
				XiSch += ((count[j] - est) * (count[j] - est)) / est;
			}
			if (XiSch < XiSquareDistribution) {
				XiSquareDistribution = XiSch;
				criticalImportance = criticalImpDop;
				resT = t;
				intervalsCount = n;
				resPs = probs;
				resCount = count;
			}
			int k = resT + 1;
			cout << "\n > Максимальная длина интервала (t): " << resT;
			cout << "\n > Количество интервалов (n): " << intervalsCount;
			cout << "\n > Границы: alpha = " << alpha << " beta = " << beta;
			cout << "\n > Пересчитанные вероятности Pr и Pt: \n";
			cout << "   ";
			for (int j = 0; j < resPs.size(); j++)
			cout << resPs[j] << " ";
			cout << "\n > Подсчитанные значения интервалов длиной 0, 1, ..., t - 1 и >= t (t = " << resT << "): \n";
			cout << "   ";
			for (int j = 0; j < resCount.size(); j++)
			cout << resCount[j] << " ";
			cout << "\n > Критическое значение хи-квадрат для " << k << " степеней свобод: " << criticalImportance;
			cout << "\n > Значение критерия хи-квадрат с " << k << " степенью свободы: " << round(XiSquareDistribution * 1000) / 1000 << "\n";
			if (0 < XiSquareDistribution && XiSquareDistribution < criticalImportance) {
				cout << "\n[+++] Критерий пройден\n";
				return true;
			}
			else {
				cout << "\n[---] Критерий не пройден\n";
				return false;
			}
		}
	\end{minted}
		
	\subsubsection{Критерий разбиений}
	
	Описание критерия:
	
	В общем случае критерия разбиений рассматриваются $n$ групп $k$ последовательных чисел, и подсчитывается число групп из $k$ чисел с $r$ различными числами. Затем применяется хи-квадрат критерий, в котором используются вероятности того, что в группе $r$ различных чисел
			\begin{center}
				$p_r = \dfrac{d (d - 1) \dots (d - r + 1)}{d^k} \left\{ \substack{k\\r} \right\}$.
			\end{center}
	
	Здесь $\left\{ \substack{k\\r} \right\}$ = $S(n, k)$ – числа Стирлинга, задающие число способов разбиения множества из $n$ элементов на $k$ непересекающихся подмножеств, которые можно вычислить по формуле:
				\begin{center}
					$S(n, k) = \dfrac{1}{k!} \sum_{j = 0}^{k} (-1)^{k + j} \left( \substack{k\\j} \right) j^n$,
					
					$\left( \substack{k\\j} \right) = \dfrac{k!}{j! (k - j)!}$.
				\end{center}

	Так как вероятности $p_r$ очень малы, когда $r$ = 1 или 2, следует, перед применением критерия хи-квадрат, объединить несколько категорий, имеющих малые вероятности в одну.
	Чтобы получить формулу для $p_r$, следует подсчитать, сколько  $d^k$ групп из $k$ чисел, расположенных между $0$ и $d - 1$, имеют точно $r$ различных элементов, и разделить это число на $d^k$. 
	
	Приведем результаты выполнения программы для каждого из генераторов:
	
	\begin{figure}[H]
		\centering
		\includegraphics[width=0.9\textwidth]{RazLc}
		\caption{Результаты выполнения программы для генератора lc}
		\label{fig:RazLc}
	\end{figure}
	
	\begin{figure}[H]
		\centering
		\includegraphics[width=0.9\textwidth]{RazAdd}
		\caption{Результаты выполнения программы для генератора add}
		\label{fig:RazAdd}
	\end{figure}
	
	\begin{figure}[H]
		\centering
		\includegraphics[width=0.9\textwidth]{Raz5p}
		\caption{Результаты выполнения программы для генератора 5p}
		\label{fig:Raz5p}
	\end{figure}	
	
	\begin{figure}[H]
		\centering
		\includegraphics[width=0.9\textwidth]{RazLfsr}
		\caption{Результаты выполнения программы для генератора lfsr}
		\label{fig:RazLfsr}
	\end{figure}
	
	\begin{figure}[H]
		\centering
		\includegraphics[width=0.9\textwidth]{RazNfsr}
		\caption{Результаты выполнения программы для генератора nfsr}
		\label{fig:RazNfsr}
	\end{figure}
	
	\begin{figure}[H]
		\centering
		\includegraphics[width=0.9\textwidth]{RazMt}
		\caption{Результаты выполнения программы для генератора mt}
		\label{fig:RazMt}
	\end{figure}		
	
	\begin{figure}[H]
		\centering
		\includegraphics[width=0.9\textwidth]{RazRc4}
		\caption{Результаты выполнения программы для генератора rc4}
		\label{fig:RazRc4}
	\end{figure}
	
	\begin{figure}[H]
		\centering
		\includegraphics[width=0.9\textwidth]{RazRsa}
		\caption{Результаты выполнения программы для генератора rsa}
		\label{fig:RazRsa}
	\end{figure}
	
	\begin{figure}[H]
		\centering
		\includegraphics[width=0.9\textwidth]{RazBbs}
		\caption{Результаты выполнения программы для генератора bbs}
		\label{fig:RazBbs}
	\end{figure}
	
	Исходный код программы:
	
	\begin{minted}[fontsize=\footnotesize]{cpp}
		bool checkPartitionCriterion(vector <double> allElems) {
			int d = 8;
			int k = 5;
			vector <int> resFives(6, 0);
			int elemsCount = allElems.size();
			while (elemsCount % 5 != 0)
			elemsCount = elemsCount - 1;
			int sGr = elemsCount / k;
			for (int i = 0; i < sGr; i++) {
				set <int> dopGr;
				for (int j = i * k; j < (i + 1) * k; j++)
				dopGr.insert(floor(allElems[j] * d));
				resFives[dopGr.size()] += 1;
			}
			double criticalImportance = 9.48773;
			vector <double> fivePs = { 0, 0.0002, 0.0256, 0.2563, 0.5127, 0.2051 };
			double XiSquareDistribution = 0;
			for (int i = 1; i < resFives.size(); i++)
			XiSquareDistribution = XiSquareDistribution + ((resFives[i] - fivePs[i] * sGr) * (resFives[i] - fivePs[i] * sGr)) / (fivePs[i] * sGr);
			XiSquareDistribution = round(XiSquareDistribution * 1000) / 1000;
			cout << "\n > Параметр d: " << d;
			cout << "\n > Параметр k: " << k;
			for (int i = 1; i < resFives.size(); i++) {
				cout << "   Количество пятерок, где " << i << " различных значений: ";
				cout << resFives[i] << "\n";
			}
			cout << "\n > Теоретические вероятности для каждой из пятерок: ";
			for (int i = 1; i < fivePs.size(); i++)
			cout << "   " << i << ": " << fivePs[i] << "\n";
			cout << "\n > Критическое значение хи-квадрат для " << k - 1 << " степеней свобод: " << criticalImportance;
			cout << "\n > Значение критерия хи-квадрат с " << k - 1 << " степенью свободы: " << XiSquareDistribution << "\n";
			if (0 < XiSquareDistribution && XiSquareDistribution < criticalImportance) {
				cout << "\n[+++] Критерий пройден\n";
				return true;
			}
			else {
				cout << "\n[---] Критерий не пройден\n";
				return false;
			}
		}
	\end{minted}
	
	\subsubsection{Критерий перестановок}
	
	Описание критерия:

	Последовательность $X^m=(x_1, x_2, \dots, x_m)$ разбивается на $n$ групп по $t$ элементов в каждой:
			\begin{center}
				$u_j = (x_{jt}, x_{jt+1}, \dots, x_{jt+t-1} ), 0 \leq j < n$.
			\end{center}
	
	Элементы в каждой группе можно упорядочивать $t!$ различными способами. Подсчитывается число групп с любым возможным порядком и применяется хи-квадрат критерий с $k$ = $t!$ возможными категориями и вероятностью $\dfrac{1}{t!}$ для каждой категории. В этом критерии предполагается, что $x_s$ не могут быть равны между собой. В реализации берется $t$ = 4. \
	
	Приведем результаты выполнения программы для каждого из генераторов:
	
	\begin{figure}[H]
		\centering
		\includegraphics[width=0.9\textwidth]{PerLc}
		\caption{Результаты выполнения программы для генератора lc}
		\label{fig:PerLc}
	\end{figure}
	
	\begin{figure}[H]
		\centering
		\includegraphics[width=0.9\textwidth]{PerAdd}
		\caption{Результаты выполнения программы для генератора add}
		\label{fig:PerAdd}
	\end{figure}
	
	\begin{figure}[H]
		\centering
		\includegraphics[width=0.9\textwidth]{Per5p}
		\caption{Результаты выполнения программы для генератора 5p}
		\label{fig:Per5p}
	\end{figure}	
	
	\begin{figure}[H]
		\centering
		\includegraphics[width=0.9\textwidth]{PerLfsr}
		\caption{Результаты выполнения программы для генератора lfsr}
		\label{fig:PerLfsr}
	\end{figure}
	
	\begin{figure}[H]
		\centering
		\includegraphics[width=0.9\textwidth]{PerNfsr}
		\caption{Результаты выполнения программы для генератора nfsr}
		\label{fig:PerNfsr}
	\end{figure}
	
	\begin{figure}[H]
		\centering
		\includegraphics[width=0.9\textwidth]{PerMt}
		\caption{Результаты выполнения программы для генератора mt}
		\label{fig:PerMt}
	\end{figure}		
	
	\begin{figure}[H]
		\centering
		\includegraphics[width=0.9\textwidth]{PerRc4}
		\caption{Результаты выполнения программы для генератора rc4}
		\label{fig:PerRc4}
	\end{figure}
	
	\begin{figure}[H]
		\centering
		\includegraphics[width=0.9\textwidth]{PerRsa}
		\caption{Результаты выполнения программы для генератора rsa}
		\label{fig:PerRsa}
	\end{figure}
	
	\begin{figure}[H]
		\centering
		\includegraphics[width=0.9\textwidth]{PerBbs}
		\caption{Результаты выполнения программы для генератора bbs}
		\label{fig:PerBbs}
	\end{figure}
	
	Исходный код программы:
	
	\begin{minted}[fontsize=\footnotesize]{cpp}
		int dopPermutationFunction(vector <double> permVec) {
			int r = 4;
			int fRes = 0;
			while (r > 0) {
				int s = 0;
				double maxV = -10.0;
				for (int i = 0; i < permVec.size(); i++)
				if (maxV < permVec[i]) {
					maxV = permVec[i];
					s = i;
				}
				s++;
				fRes = r * fRes + s - 1;
				double dopV = permVec[r - 1];
				permVec[r - 1] = permVec[s - 1];
				permVec[s - 1] = dopV;
				permVec.resize(permVec.size() - 1);
				r--;
			}
			return fRes;
		}
		
		bool checkPermutationsCriterion(vector <double> allElems) {
			int t = 4;
			int myT = 1;
			for (int i = 1; i <= t; i++)
			myT *= i;
			int sizeinVals = allElems.size();
			while (sizeinVals % 4 != 0)
			sizeinVals = sizeinVals - 1;
			double criticalImportance = 36.41503;
			int nGr = sizeinVals / t;
			vector <int> resCategories(myT, 0);
			for (int i = 0; i < nGr; i++) {
				vector <double> cGr;
				set <double> dopGrc;
				for (int j = i * t; j < (i + 1) * t; j++) {
					cGr.push_back(allElems[j]);
					dopGrc.insert(allElems[j]);
				}
				if (dopGrc.size() != t)
				continue;
				else
				resCategories[dopPermutationFunction(cGr)] += 1;
			}
			double expectedP = round((1.0 / myT) * 1000) / 1000;
			double XiSquareDistribution = 0.0;
			double expectedCategoriesValue = round((expectedP * nGr) * 1000) / 1000;
			cout << "\n > Параметр t: " << t;
			cout << "\n > Теоретическое количество попаданий в каждую категорию: " << expectedCategoriesValue;
			cout << "\n > Вычисленное распределение по частотам: ";
			for (int i = 0; i < resCategories.size(); i++)
			cout << resCategories[i] << " ";
			cout << "\n > Теоретическое значение вероятности для каждой категории: " << expectedP;
			for (int i = 0; i < myT; i++)
			XiSquareDistribution = XiSquareDistribution + ((resCategories[i] - expectedCategoriesValue) * (resCategories[i] - expectedCategoriesValue)) / expectedCategoriesValue;
			XiSquareDistribution = round(XiSquareDistribution * 1000) / 1000;
			cout << "\n > Критическое значение хи-квадрат для " << myT << " степеней свобод: " << criticalImportance;
			cout << "\n > Значение критерия хи-квадрат с " << myT << " степенью свободы: " << XiSquareDistribution << "\n";
			if (0 < XiSquareDistribution && XiSquareDistribution < criticalImportance) {
				cout << "\n[+++] Критерий пройден\n";
				return true;
			}
			else {
				cout << "\n[---] Критерий не пройден\n";
				return false;
			}
		}
	\end{minted}
		
	\subsubsection{Критерий монотонности}
	
	Описание критерия:
	
	Последовательность можно проверить на предмет равномерности распре-
	деления монотонных серий чисел.
	
	Суть метода в том, чтобы проверить длины всех восходящих серий $c_i$ в последовательности и подсчитать для них статистику.
	
	Для решения проблемы чередования длинных серий с короткими сериями можно сделать следующее: 
	
	\begin{enumerate}
		\item «Выбрасываем» элемент последовательности, который следует непосредственно за серией.
		\item Если $x_j$  больше $x_{j+1}$, то начнем следующую серию с $x_{j+2}$.
		\item Мы получаем серии, длины которых независимы и, поэтому, можно использовать критерий хи-квадрат.
	\end{enumerate}
	
	Приведем результаты выполнения программы для каждого из генераторов:
	
	\begin{figure}[H]
		\centering
		\includegraphics[width=0.9\textwidth]{MonLc}
		\caption{Результаты выполнения программы для генератора lc}
		\label{fig:MonLc}
	\end{figure}
	
	\begin{figure}[H]
		\centering
		\includegraphics[width=0.9\textwidth]{MonAdd}
		\caption{Результаты выполнения программы для генератора add}
		\label{fig:MonAdd}
	\end{figure}
	
	\begin{figure}[H]
		\centering
		\includegraphics[width=0.9\textwidth]{Mon5p}
		\caption{Результаты выполнения программы для генератора 5p}
		\label{fig:Mon5p}
	\end{figure}	
	
	\begin{figure}[H]
		\centering
		\includegraphics[width=0.9\textwidth]{MonLfsr}
		\caption{Результаты выполнения программы для генератора lfsr}
		\label{fig:MonLfsr}
	\end{figure}
	
	\begin{figure}[H]
		\centering
		\includegraphics[width=0.9\textwidth]{MonNfsr}
		\caption{Результаты выполнения программы для генератора nfsr}
		\label{fig:MonNfsr}
	\end{figure}
	
	\begin{figure}[H]
		\centering
		\includegraphics[width=0.9\textwidth]{MonMt}
		\caption{Результаты выполнения программы для генератора mt}
		\label{fig:MonMt}
	\end{figure}		
	
	\begin{figure}[H]
		\centering
		\includegraphics[width=0.9\textwidth]{MonRc4}
		\caption{Результаты выполнения программы для генератора rc4}
		\label{fig:MonRc4}
	\end{figure}
	
	\begin{figure}[H]
		\centering
		\includegraphics[width=0.9\textwidth]{MonRsa}
		\caption{Результаты выполнения программы для генератора rsa}
		\label{fig:MonRsa}
	\end{figure}
	
	\begin{figure}[H]
		\centering
		\includegraphics[width=0.9\textwidth]{MonBbs}
		\caption{Результаты выполнения программы для генератора bbs}
		\label{fig:MonBbs}
	\end{figure}
	
	Исходный код программы:
	
	\begin{minted}[fontsize=\footnotesize]{cpp}
		bool checkMonotonicityCriterion(vector <double> allElems) {
			vector <pair <int, int>> categor;
			vector <double> ojid;
			double XiSquareDistribution = 0.0;
			int posl = 1, i = 1;
			while (i < allElems.size() - 1) {
				if (allElems[i] < allElems[i + 1])
				posl += 1;
				else {
					int num = -1;
					if (categor.size() != 0) {
						for (int j = 0; j < categor.size(); j++) {
							if (categor[j].first == posl) {
								num = j;
								break;
							}
						}
						if (num == -1)
						categor.push_back(make_pair(posl, 1));
						else
						categor[num].second += 1;
					}
					else
					categor.push_back(make_pair(posl, 1));
					posl = 1;
					i = i + 1;
				}
				i += 1;
			}
			int sumS = 0;
			for (int j = 0; j < categor.size(); j++)
			sumS = sumS + categor[j].second;
			int num_categs = categor.size();
			double criticalImportance = InterVector[num_categs - 1];
			int maximal = -1;
			for (int j = 0; j < categor.size(); j++)
			if (categor[j].first > maximal)
			maximal = categor[j].first;
			long long fact = 1;
			for (int j = 1; j < maximal + 1; j++) {
				fact *= j;
				ojid.push_back(1.0 / fact - 1.0 / (fact * (j + 1)));
			}
			for (int j = 0; j < num_categs; j++)
			ojid[j] = (sumS * ojid[j]);
			for (int j = 0; j < num_categs; j++)
			XiSquareDistribution = XiSquareDistribution + round((((categor[j].second - ojid[categor[j].first - 1]) * (categor[j].second - ojid[categor[j].first - 1])) / ojid[categor[j].first - 1]) * 1000) / 1000;
			cout << "\n > Теоретические значения длин серий: ";
			for (int j = 0; j < ojid.size(); j++)
			cout << round(ojid[j]) << " ";
			cout << "\n > Вычисленные значения длин серий: ";
			set <pair <int, int>> need;
			for (int j = 0; j < categor.size(); j++)
			need.insert(categor[j]);
			for (pair <int, int> a : need)
			cout << a.first << ": " << a.second << " ";
			cout << "\n > Критическое значение хи-квадрат для " << num_categs << " степеней свобод: " << criticalImportance;
			cout << "\n > Значение критерия хи-квадрат с " << num_categs << " степенью свободы: " << XiSquareDistribution << "\n";
			if (0 < XiSquareDistribution && XiSquareDistribution < criticalImportance) {
				cout << "\n[+++] Критерий пройден\n";
				return true;
			}
			else {
				cout << "\n[---] Критерий не пройден\n";
				return false;
			}
			
		}
	\end{minted}
		
	\subsubsection{Критерий конфликтов}
	
	Описание критерия:
	
	Предположим, нам нужно оценить последовательность случайных чисел, в которой число величин в последовательности намного меньше числа категорий. В этом случае критерий хи-квадрат не применим, но можно использовать критерий конфликтов. 
	
	Предположим, что у нас m урн и n шаров, причем m значительно больше $n$. Если разместить шары в урнах наугад, то некоторые урны останутся пустыми, а в некоторых будет более одного шара. Когда в одну урну попадает больше одного шара, то говорят, что произошел «конфликт». Критерий конфликтов состоит в подсчете и оценке количества конфликтов.
	
	Рассмотрим пример, когда $m = 2^{20}$, а $n = 2^{14}$. В среднем, число урн, приходящихся на один шар – 64. Вероятность того, что в конкретную урну попадет ровно $k$ шаров, равна
	
			\begin{center}
				$p_k = \left( \substack{n\\k} \right) m^{-k} (1-m^{-1})^{n-k}$,
			\end{center}
		
	отсюда, среднее число конфликтов в урне вычисляется по формуле
	
			\begin{center}
				$\sum_{k \leq 1} (k - 1) p_k = \sum_{k \leq 1} k p_k - \sum_{k \leq 1} p_k = \dfrac{n}{m} - 1 + p_0$.
			\end{center}
	
	Так как $p_0 = (1 - m^{-1} )^n = 1-n m^{-1} + \left( \substack{n\\2} \right) m^(-2)$ – маленькое число, получим, что общее среднее число конфликтов во всех $m$ урнах намного меньше 
	
				\begin{center}
					$\dfrac{n^2}{2 m} = 128$.
				\end{center}
	
	Приведем результаты выполнения программы для каждого из генераторов:
	
	\begin{figure}[H]
		\centering
		\includegraphics[width=0.9\textwidth]{KonLc}
		\caption{Результаты выполнения программы для генератора lc}
		\label{fig:KonLc}
	\end{figure}
	
	\begin{figure}[H]
		\centering
		\includegraphics[width=0.9\textwidth]{KonAdd}
		\caption{Результаты выполнения программы для генератора add}
		\label{fig:KonAdd}
	\end{figure}
	
	\begin{figure}[H]
		\centering
		\includegraphics[width=0.9\textwidth]{Kon5p}
		\caption{Результаты выполнения программы для генератора 5p}
		\label{fig:Kon5p}
	\end{figure}	
	
	\begin{figure}[H]
		\centering
		\includegraphics[width=0.9\textwidth]{KonLfsr}
		\caption{Результаты выполнения программы для генератора lfsr}
		\label{fig:KonLfsr}
	\end{figure}
	
	\begin{figure}[H]
		\centering
		\includegraphics[width=0.9\textwidth]{KonNfsr}
		\caption{Результаты выполнения программы для генератора nfsr}
		\label{fig:KonNfsr}
	\end{figure}
	
	\begin{figure}[H]
		\centering
		\includegraphics[width=0.9\textwidth]{KonMt}
		\caption{Результаты выполнения программы для генератора mt}
		\label{fig:KonMt}
	\end{figure}		
	
	\begin{figure}[H]
		\centering
		\includegraphics[width=0.9\textwidth]{KonRc4}
		\caption{Результаты выполнения программы для генератора rc4}
		\label{fig:KonRc4}
	\end{figure}
	
	\begin{figure}[H]
		\centering
		\includegraphics[width=0.9\textwidth]{KonRsa}
		\caption{Результаты выполнения программы для генератора rsa}
		\label{fig:KonRsa}
	\end{figure}
	
	\begin{figure}[H]
		\centering
		\includegraphics[width=0.9\textwidth]{KonBbs}
		\caption{Результаты выполнения программы для генератора bbs}
		\label{fig:KonBbs}
	\end{figure}
	
	Исходный код программы:
	
	\begin{minted}[fontsize=\footnotesize]{cpp}
		vector <pair <int, double>> ConflictCriterionPercentPoints(int m, int n) {
			vector <double> auxiliaryTableT = { 0.01, 0.05, 0.25, 0.50, 0.75, 0.95, 0.99, 1.0 };
			vector <double> auxiliaryA(n + 1, 0.0);
			vector <pair <int, double>> conflictsResP;
			auxiliaryA[1] = 1.0;
			int j0 = 1;
			int j1 = 1;
			for (int i = 0; i < n - 1; i++) {
				j1++;
				for (int j = j1; j > j0 - 1; j--) {
					double jm = j / (m * 1.0);
					auxiliaryA[j] = jm * auxiliaryA[j] + (1.0 + 1.0 / m - jm) * auxiliaryA[j - 1];
					if (auxiliaryA[j] < 1e-20) {
						auxiliaryA[j] = 0.0;
						if (j == j1) {
							j1--;
							continue;
						}
						if (j == j0)
						j0++;
					}
				}
			}
			int t = 0;
			int j = j0 - 1;
			double p = 0.0;
			while (t != auxiliaryTableT.size() - 1) {
				while (p <= auxiliaryTableT[t]) {
					j++;
					p += auxiliaryA[j];
				}
				conflictsResP.push_back(make_pair(n - j - 1, round((1 - p) * 1000) / 1000));
				t++;
			}
			reverse(conflictsResP.begin(), conflictsResP.end());
			return conflictsResP;
			
		}
		
		bool checkConflictCriterion(vector <double> allElems) {
			srand((unsigned int)time(0));
			double eps = 1e-20;
			vector <vector <pair<int, double>>> appropriatePercentPoints;
			vector <vector <int>> appropriatePointsTable;
			for (int bigSize = 8; bigSize < 21; bigSize++) {
				int nParameter = allElems.size() / bigSize;
				for (int dParameter = 2; dParameter < 9; dParameter++) {
					int j = 0;
					vector <int> normSeq;
					while (j < allElems.size()) {
						if (allElems[j] == 1.0)
						allElems[j] = 0.965;
						normSeq.push_back((floor(allElems[j] * dParameter)));
						j++;
					}
					set <vector <int>> words;
					int nDimension = 0;
					for (int jI = 0; jI < nParameter; jI++) {
						vector <int> dopSliced;
						for (int i = jI * bigSize; i < (jI + 1) * bigSize; i++) {
							dopSliced.push_back(normSeq[i]);
						}
						auto search = words.find(dopSliced);
						if (search == words.end())
						words.insert(dopSliced);
						else
						nDimension += 1;
					}
					for (int mParameter = 16; mParameter < 129; mParameter++) {
						int  mV = nParameter * mParameter;
						vector <pair <int, double>> confs_et_probs = ConflictCriterionPercentPoints(mV, nParameter);
						if (nDimension == 0 || confs_et_probs[0].first == -1 || confs_et_probs[0].first == 0)
						continue;
						if (confs_et_probs[2].first <= nDimension && nDimension <= confs_et_probs[confs_et_probs.size() - 2].first) {
							appropriatePercentPoints.push_back(confs_et_probs);
							vector <int> prom = { nDimension, bigSize, nParameter, dParameter, mParameter, mV };
							appropriatePointsTable.push_back(prom);
						}
					}
				}
			}
			int appropriatePercentPointsCount = appropriatePercentPoints.size();
			set <int> setRand;
			for (int j = 0; j < 5; j++) {
				
				int randInt = (double)(rand()) / RAND_MAX * appropriatePercentPointsCount - 1;
				auto search1 = setRand.find(randInt);
				if (search1 == setRand.end())
				setRand.insert(randInt);
				else {
					while (search1 == setRand.end()) {
						randInt = (double)(rand()) / RAND_MAX * appropriatePercentPointsCount - 1;
						search1 = setRand.find(randInt);
					}
				}
			}
			cout << "\n > Число параметров, при которых представленная последовательность удовлетворяет критерию конфликтов: " << appropriatePercentPointsCount;
			cout << "\n > Приведем случай, которая параметры подходят:";
			cout << "\n     Размерность вектора Vj: " << appropriatePointsTable[0][1];
			cout << "\n     Количество векторов: " << appropriatePointsTable[0][2];
			cout << "\n     Значение m: " << appropriatePointsTable[0][5];
			cout << "\n     Множитель для m: " << appropriatePointsTable[0][4];
			cout << "\n     Параметр нормирования d: " << appropriatePointsTable[0][3];
			cout << "\n     Количество возникших конфликтов: " << appropriatePointsTable[0][0];
			cout << "\n     Таблица процентных точек: ";
			for (int k = 0; k < appropriatePercentPoints[0].size(); k++) {
				cout << "(" << appropriatePercentPoints[0][k].first << ", " << appropriatePercentPoints[0][k].second << ") ";
			}
			cout << "\n";
			if (appropriatePercentPointsCount > 0) {
				cout << "\n[+++] Критерий пройден\n";
				return true;
			}
			else {
				cout << "\n[---] Критерий не пройден\n";
				return false;
			}
		}
	\end{minted}
	
	\subsubsection{Результаты проверки всех критериев для последовательностей, сгенерированных каждым генератором}
	
	В следующей таблице представлены результаты проверки для всех сгенерированных последовательностей по каждому из рассмотренных выше критериев. Зеленый цвет - критерий пройден, красный - не пройден.
	
	\begin{center}
		\begin{tabular}{|c|c|c|c|c|c|c|c|}
			\hline
			$ $ & $\chi^2$ & серий & интервал. & разбиений & перест. & монот. & конфл. \\
			\hline
			lc & \cellcolor[RGB]{71,167,106} 3.2297 & \cellcolor[RGB]{71,167,106} 3.353 & \cellcolor[RGB]{71,167,106} 13.313 & \cellcolor[RGB]{71,167,106} 2.037 & \cellcolor[RGB]{155,45,48} 119.287 & \cellcolor[RGB]{155,45,48} 46.48 & \cellcolor[RGB]{71,167,106} 148 \\
			\hline
			add & \cellcolor[RGB]{155,45,48} 24.7563 & \cellcolor[RGB]{71,167,106} 15.019 & \cellcolor[RGB]{71,167,106} 9.321 & \cellcolor[RGB]{71,167,106} 9.402 & \cellcolor[RGB]{71,167,106} 18.99 & \cellcolor[RGB]{71,167,106} 2.279 & \cellcolor[RGB]{71,167,106} 106 \\
			\hline
			5p & \cellcolor[RGB]{71,167,106} 9.5378 & \cellcolor[RGB]{155,45,48} 65.929 & \cellcolor[RGB]{71,167,106} 12.985 & \cellcolor[RGB]{71,167,106} 3.94 & \cellcolor[RGB]{155,45,48} 131.638 & \cellcolor[RGB]{155,45,48} 3991940 & \cellcolor[RGB]{71,167,106} 378 \\
			\hline
			lfsr & \cellcolor[RGB]{71,167,106} 12.902 & \cellcolor[RGB]{155,45,48} 2097.11 & \cellcolor[RGB]{155,45,48} 30.225 & \cellcolor[RGB]{155,45,48} 52.143 & \cellcolor[RGB]{155,45,48} 481.057 & \cellcolor[RGB]{155,45,48} 371.465 & \cellcolor[RGB]{71,167,106} 211 \\
			\hline
			nfsr & \cellcolor[RGB]{71,167,106} 7.7395 & \cellcolor[RGB]{155,45,48} 3947.99 & \cellcolor[RGB]{155,45,48} 358.885 & \cellcolor[RGB]{155,45,48} 677.402 & \cellcolor[RGB]{155,45,48} 1584.28 & \cellcolor[RGB]{155,45,48} 43547900 & \cellcolor[RGB]{71,167,106} 333 \\
			\hline
			mt & \cellcolor[RGB]{71,167,106} 13.916 & \cellcolor[RGB]{71,167,106} 12.808 & \cellcolor[RGB]{71,167,106} 5.571 & \cellcolor[RGB]{71,167,106} 2.233 & \cellcolor[RGB]{71,167,106} 20.038 & \cellcolor[RGB]{71,167,106} 6.745 & \cellcolor[RGB]{71,167,106} 115 \\
			\hline
			rc4 & \cellcolor[RGB]{155,45,48} 822.272 & \cellcolor[RGB]{155,45,48} 135.25 & \cellcolor[RGB]{155,45,48} 30.929 & \cellcolor[RGB]{71,167,106} 4.392 & \cellcolor[RGB]{71,167,106} 26.81 & \cellcolor[RGB]{71,167,106} 10.65 & \cellcolor[RGB]{71,167,106} 130 \\
			\hline
			rsa & \cellcolor[RGB]{155,45,48} 164.639 & \cellcolor[RGB]{155,45,48} 57.526 & \cellcolor[RGB]{155,45,48} 29.857 & \cellcolor[RGB]{71,167,106} 3.496 & \cellcolor[RGB]{155,45,48} 11.238 & \cellcolor[RGB]{71,167,106} 18.575 & \cellcolor[RGB]{71,167,106} 87 \\
			\hline
			bbs & \cellcolor[RGB]{71,167,106} 17.7563 & \cellcolor[RGB]{71,167,106} 16.006 & \cellcolor[RGB]{71,167,106} 6.135 & \cellcolor[RGB]{71,167,106} 4.856 & \cellcolor[RGB]{71,167,106} 18.029 & \cellcolor[RGB]{71,167,106} 1.422 & \cellcolor[RGB]{71,167,106} 99 \\
			\hline
		\end{tabular}
	\end{center}
	
	
	\newpage
	\appendix
	\section{Исходный код программы проверки критериев}
	
	\begin{minted}[fontsize=\footnotesize]{cpp}
		#define _USE_MATH_DEFINES
		
		#include <iostream>
		#include <string>
		#include <vector>
		#include <fstream>
		#include <iomanip>
		#include <bitset>
		#include <sstream>
		#include <deque>
		#include <cmath>
		#include <set>
		
		using namespace std;
		
		string findInStr(string const& str, int n) {
			if (str.length() < n) {
				return str;
			}
			return str.substr(0, n);
		}
		
		void helpFunc() {
			cout << "\n\n/f:<имя_файла> - имя файла с входной последовательностью";
			cout << "\n\n/h – информация о допустимых параметрах командной строки программы.\n";
		}
		
		
		double calculateExpValue(vector <double> allElems) {
			double elemsCount = allElems.size();
			double elemsSum = 0;
			for (int i = 0; i < allElems.size(); i++) {
				elemsSum += allElems[i];
			}
			return elemsSum / elemsCount;
		}
		
		double calculateStandartDeviation(vector <double> allElems, double expValue) {
			double elemsCount = allElems.size();
			double elemsSum = 0;
			for (int i = 0; i < allElems.size(); i++) {
				elemsSum = elemsSum + (allElems[i] - expValue) * (allElems[i] - expValue);
			}
			return sqrt(elemsSum / elemsCount);
		}
		
		double calculateExpValueError(double expValue) {
			return abs(expValue - 0.5) / expValue;
		}
		
		double calculateStandartDeviationError(double standartDeviation) {
			double sqrt1_12 = sqrt(1.0 / 12);
			return abs(standartDeviation - sqrt1_12) / standartDeviation;
		}
		
		bool checkXiSquareCriterion(vector <double> allElems) {
			double maxElem = -1.0, minElem = 1.0;
			int elemsCount = allElems.size();
			for (int i = 0; i < elemsCount; i++)
			{
				double checkElem = allElems[i];
				if (checkElem > maxElem)
				maxElem = checkElem;
				if (checkElem < minElem)
				minElem = checkElem;
			}
			vector <pair <double, double>> allIntervals;
			int intervalsCount = ceil(1 + 1.4 * log(elemsCount));
			double IntS = (round(((maxElem - minElem) / intervalsCount) * 10000)) / 10000;
			for (int i = 0; i < intervalsCount; i++) {
				double a_i = minElem + IntS * i;
				double b_i = minElem + IntS * (i + 1);
				allIntervals.push_back(make_pair(a_i, b_i));
			}
			double expectedNumOfHits = round((elemsCount / intervalsCount) * 1000) / 1000;
			vector <int> realDistribution(intervalsCount, 0);
			for (int i = 0; i < elemsCount; i++)
			for (int j = 0; j < intervalsCount; j++)
			if (allIntervals[j].first <= allElems[i] && allElems[i] < allIntervals[j].second) {
				realDistribution[j]++;
				break;
			}
			double XiSquareDistribution = 0;
			for (int i = 0; i < intervalsCount; i++)
			XiSquareDistribution = XiSquareDistribution + ((realDistribution[i] - expectedNumOfHits) * (realDistribution[i] - expectedNumOfHits)) / expectedNumOfHits;
			XiSquareDistribution = round(XiSquareDistribution * 10000) / 10000;
			double criticalImportance = 22.36203;
			cout << "\n > Количество интервалов: " << intervalsCount;
			int k = intervalsCount - 1;
			cout << "\n > Ожидаемое число попаданий в интервалы: " << expectedNumOfHits;
			cout << "\n > Число попаданий в каждый из интервалов: ";
			for (int i = 0; i < intervalsCount; i++)
			cout << i + 1 << ": " << realDistribution[i] << endl;
			cout << "\n > Критическое значение хи-квадрат для " << k << " степеней свобод: " << criticalImportance;
			cout << "\n > Значение критерия хи-квадрат с " << k << " степенью свободы: " << XiSquareDistribution << "\n";
			if (0 < XiSquareDistribution && XiSquareDistribution < criticalImportance) {
				cout << "\n[+++] Критерий пройден\n";
				return true;
			}
			else {
				cout << "\n[---] Критерий не пройден\n";
				return false;
			}
		}
		
		bool checkSeriesCriterion(vector <double> allElems) {
			int d = 4;
			vector <vector <int>> nPairs(d);
			for (int i = 0; i < d; i++)
			nPairs[i].resize(d, 0);
			int k = d * d;
			vector <pair <int, int>> pairCategories;
			int elemsCount = allElems.size();
			if (elemsCount % 2 != 0)
			elemsCount--;
			for (int j = 0; j < elemsCount; j+=2)
			{
				double leftI = floor(allElems[j] * d);
				double rightI = floor(allElems[j + 1] * d);
				pairCategories.push_back(make_pair(leftI, rightI));
			}
			double criticalImportance = 26.29623;
			double expectedNumOfHits = round((elemsCount / (2 * k)) * 1000) / 1000;
			for (int i = 0; i < pairCategories.size(); i++) {
				nPairs[pairCategories[i].first][pairCategories[i].second] += 1;
			}
			double XiSquareDistribution = 0;
			for (int i = 0; i < nPairs.size(); i++)
			for (int x = 0; x < nPairs[i].size(); x++)
			XiSquareDistribution += ((nPairs[i][x] - expectedNumOfHits) * (nPairs[i][x] - expectedNumOfHits)) / expectedNumOfHits;
			XiSquareDistribution = round(XiSquareDistribution * 1000) / 1000;
			cout << "\n > Параметр d: " << d;
			cout << "\n > Ожидаемое количество чисел в каждой категории: " << expectedNumOfHits;
			cout << "\n > Количество чисел в каждой категории: ";
			int chCount = 1;
			for (int i = 0; i < nPairs.size(); i++) {
				for (int x = 0; x < nPairs[i].size(); x++) {
					chCount++;
					cout << "   " << chCount << ": " << nPairs[i][x] << "\n";
				}
				chCount++;
			}
			cout << "\n > Критическое значение хи-квадрат для " << k << " степеней свобод: " << criticalImportance;
			cout << "\n > Значение критерия хи-квадрат с " << k << " степенью свободы: " << XiSquareDistribution << "\n";
			if (0 < XiSquareDistribution && XiSquareDistribution < criticalImportance) {
				cout << "\n[+++] Критерий пройден\n";
				return true;
			}
			else {
				cout << "\n[---] Критерий не пройден\n";
				return false;
			}
		}
		
		double powDouble(double c, int stepen) {
			if (stepen == 0)
			return 1.0;
			if (stepen == 1)
			return c;
			double res = c;
			for (int i = 2; i <= stepen; i++)
			res = res * c;
			return res;
		}
		
		vector < double > InterVector = { 3.84146 , 5.99146, 7.81473, 9.48773, 11.07050, 12.59159, 14.06714, 15.50731, 16.91898, 18.30704, 19.67514, 21.02607, 22.36203, 23.68479, 24.99579, 26.29623, 27.58711, 28.86930, 30.14353, 31.41043, 32.67057, 33.92444, 35.17246, 36.41503, 37.65248, 38.88514, 40.11327, 41.33714, 42.55697, 43.77297 };
		
		bool checkIntervalsCriterion(vector <double> allElems) {
			int t = 10, n = 1000;
			srand((unsigned int)time(0));
			double alpha = (double)(rand()) / RAND_MAX, beta = (double)(rand()) / RAND_MAX;
			while (0.2 >= abs(alpha - beta) || abs(alpha - beta) >= 0.4) {
				alpha = (double)(rand()) / RAND_MAX;
				beta = (double)(rand()) / RAND_MAX;
			}
			if (beta < alpha) {
				double dopD = alpha;
				alpha = beta;
				beta = dopD;
			}
			alpha = round(alpha * 1000) / 1000;
			beta = round(beta * 1000) / 1000;
			double p = beta - alpha;
			int s = 0;
			vector <int> count, resCount;
			double XiSquareDistribution = 100000000000.0;
			double criticalImportance = 100000000000.0;
			int resT = 100000000000;
			int intervalsCount = 100000000000;
			vector <double> resPs;
			s = 0;
			count.resize(t + 1, 0);
			int r = 0;
			for (int j = 0; j < allElems.size(); j++) {
				if (alpha <= allElems[j] && allElems[j] < beta) {
					if (r >= t)
					count[t]++;
					else
					count[r]++;
					s++;
					if (s < n)
					r = 0;
					else
					break;
				}
				else
				r++;
			}
			vector <double> probs;
			for (int j = 0; j < t; j++)
			probs.push_back(round((p * (powDouble(1.0 - p, j))) * 1000) / 1000);
			double XiSch = 0.0;
			probs.push_back(powDouble(1.0 - p, t));
			double criticalImpDop = InterVector[t];
			int qN = 0;
			for (int j = 0; j < count.size(); j++)
			qN += count[j];
			for (int j = 0; j < probs.size(); j++) {
				double est = qN * probs[j];
				if (est == 0.0)
				break;
				XiSch += ((count[j] - est) * (count[j] - est)) / est;
			}
			if (XiSch < XiSquareDistribution) {
				XiSquareDistribution = XiSch;
				criticalImportance = criticalImpDop;
				resT = t;
				intervalsCount = n;
				resPs = probs;
				resCount = count;
			}
			int k = resT + 1;
			cout << "\n > Максимальная длина интервала (t): " << resT;
			cout << "\n > Количество интервалов (n): " << intervalsCount;
			cout << "\n > Границы: alpha = " << alpha << " beta = " << beta;
			cout << "\n > Пересчитанные вероятности Pr и Pt: \n";
			cout << "   ";
			for (int j = 0; j < resPs.size(); j++)
			cout << resPs[j] << " ";
			cout << "\n > Подсчитанные значения интервалов длиной 0, 1, ..., t - 1 и >= t (t = " << resT << "): \n";
			cout << "   ";
			for (int j = 0; j < resCount.size(); j++)
			cout << resCount[j] << " ";
			cout << "\n > Критическое значение хи-квадрат для " << k << " степеней свобод: " << criticalImportance;
			cout << "\n > Значение критерия хи-квадрат с " << k << " степенью свободы: " << round(XiSquareDistribution * 1000) / 1000 << "\n";
			if (0 < XiSquareDistribution && XiSquareDistribution < criticalImportance) {
				cout << "\n[+++] Критерий пройден\n";
				return true;
			}
			else {
				cout << "\n[---] Критерий не пройден\n";
				return false;
			}
		}
		
		bool checkPartitionCriterion(vector <double> allElems) {
			int d = 8;
			int k = 5;
			vector <int> resFives(6, 0);
			int elemsCount = allElems.size();
			while (elemsCount % 5 != 0)
			elemsCount = elemsCount - 1;
			int sGr = elemsCount / k;
			for (int i = 0; i < sGr; i++) {
				set <int> dopGr;
				for (int j = i * k; j < (i + 1) * k; j++)
				dopGr.insert(floor(allElems[j] * d));
				resFives[dopGr.size()] += 1;
			}
			double criticalImportance = 9.48773;
			vector <double> fivePs = { 0, 0.0002, 0.0256, 0.2563, 0.5127, 0.2051 };
			double XiSquareDistribution = 0;
			for (int i = 1; i < resFives.size(); i++)
			XiSquareDistribution = XiSquareDistribution + ((resFives[i] - fivePs[i] * sGr) * (resFives[i] - fivePs[i] * sGr)) / (fivePs[i] * sGr);
			XiSquareDistribution = round(XiSquareDistribution * 1000) / 1000;
			cout << "\n > Параметр d: " << d;
			cout << "\n > Параметр k: " << k;
			for (int i = 1; i < resFives.size(); i++) {
				cout << "   Количество пятерок, где " << i << " различных значений: ";
				cout << resFives[i] << "\n";
			}
			cout << "\n > Теоретические вероятности для каждой из пятерок: ";
			for (int i = 1; i < fivePs.size(); i++)
			cout << "   " << i << ": " << fivePs[i] << "\n";
			cout << "\n > Критическое значение хи-квадрат для " << k - 1 << " степеней свобод: " << criticalImportance;
			cout << "\n > Значение критерия хи-квадрат с " << k - 1 << " степенью свободы: " << XiSquareDistribution << "\n";
			if (0 < XiSquareDistribution && XiSquareDistribution < criticalImportance) {
				cout << "\n[+++] Критерий пройден\n";
				return true;
			}
			else {
				cout << "\n[---] Критерий не пройден\n";
				return false;
			}
		}
		
		int dopPermutationFunction(vector <double> permVec) {
			int r = 4;
			int fRes = 0;
			while (r > 0) {
				int s = 0;
				double maxV = -10.0;
				for (int i = 0; i < permVec.size(); i++)
				if (maxV < permVec[i]) {
					maxV = permVec[i];
					s = i;
				}
				s++;
				fRes = r * fRes + s - 1;
				double dopV = permVec[r - 1];
				permVec[r - 1] = permVec[s - 1];
				permVec[s - 1] = dopV;
				permVec.resize(permVec.size() - 1);
				r--;
			}
			return fRes;
		}
		
		bool checkPermutationsCriterion(vector <double> allElems) {
			int t = 4;
			int myT = 1;
			for (int i = 1; i <= t; i++)
			myT *= i;
			int sizeinVals = allElems.size();
			while (sizeinVals % 4 != 0)
			sizeinVals = sizeinVals - 1;
			double criticalImportance = 36.41503;
			int nGr = sizeinVals / t;
			vector <int> resCategories(myT, 0);
			for (int i = 0; i < nGr; i++) {
				vector <double> cGr;
				set <double> dopGrc;
				for (int j = i * t; j < (i + 1) * t; j++) {
					cGr.push_back(allElems[j]);
					dopGrc.insert(allElems[j]);
				}
				if (dopGrc.size() != t)
				continue;
				else
				resCategories[dopPermutationFunction(cGr)] += 1;
			}
			double expectedP = round((1.0 / myT) * 1000) / 1000;
			double XiSquareDistribution = 0.0;
			double expectedCategoriesValue = round((expectedP * nGr) * 1000) / 1000;
			cout << "\n > Параметр t: " << t;
			cout << "\n > Теоретическое количество попаданий в каждую категорию: " << expectedCategoriesValue;
			cout << "\n > Вычисленное распределение по частотам: ";
			for (int i = 0; i < resCategories.size(); i++)
			cout << resCategories[i] << " ";
			cout << "\n > Теоретическое значение вероятности для каждой категории: " << expectedP;
			for (int i = 0; i < myT; i++)
			XiSquareDistribution = XiSquareDistribution + ((resCategories[i] - expectedCategoriesValue) * (resCategories[i] - expectedCategoriesValue)) / expectedCategoriesValue;
			XiSquareDistribution = round(XiSquareDistribution * 1000) / 1000;
			cout << "\n > Критическое значение хи-квадрат для " << myT << " степеней свобод: " << criticalImportance;
			cout << "\n > Значение критерия хи-квадрат с " << myT << " степенью свободы: " << XiSquareDistribution << "\n";
			if (0 < XiSquareDistribution && XiSquareDistribution < criticalImportance) {
				cout << "\n[+++] Критерий пройден\n";
				return true;
			}
			else {
				cout << "\n[---] Критерий не пройден\n";
				return false;
			}
		}
		
		bool checkMonotonicityCriterion(vector <double> allElems) {
			vector <pair <int, int>> categor;
			vector <double> ojid;
			double XiSquareDistribution = 0.0;
			int posl = 1, i = 1;
			while (i < allElems.size() - 1) {
				if (allElems[i] < allElems[i + 1])
				posl += 1;
				else {
					int num = -1;
					if (categor.size() != 0) {
						for (int j = 0; j < categor.size(); j++) {
							if (categor[j].first == posl) {
								num = j;
								break;
							}
						}
						if (num == -1)
						categor.push_back(make_pair(posl, 1));
						else
						categor[num].second += 1;
					}
					else
					categor.push_back(make_pair(posl, 1));
					posl = 1;
					i = i + 1;
				}
				i += 1;
			}
			int sumS = 0;
			for (int j = 0; j < categor.size(); j++)
			sumS = sumS + categor[j].second;
			int num_categs = categor.size();
			double criticalImportance = InterVector[num_categs - 1];
			int maximal = -1;
			for (int j = 0; j < categor.size(); j++)
			if (categor[j].first > maximal)
			maximal = categor[j].first;
			long long fact = 1;
			for (int j = 1; j < maximal + 1; j++) {
				fact *= j;
				ojid.push_back(1.0 / fact - 1.0 / (fact * (j + 1)));
			}
			for (int j = 0; j < num_categs; j++)
			ojid[j] = (sumS * ojid[j]);
			for (int j = 0; j < num_categs; j++)
			XiSquareDistribution = XiSquareDistribution + round((((categor[j].second - ojid[categor[j].first - 1]) * (categor[j].second - ojid[categor[j].first - 1])) / ojid[categor[j].first - 1]) * 1000) / 1000;
			cout << "\n > Теоретические значения длин серий: ";
			for (int j = 0; j < ojid.size(); j++)
			cout << round(ojid[j]) << " ";
			cout << "\n > Вычисленные значения длин серий: ";
			set <pair <int, int>> need;
			for (int j = 0; j < categor.size(); j++)
			need.insert(categor[j]);
			for (pair <int, int> a : need)
			cout << a.first << ": " << a.second << " ";
			cout << "\n > Критическое значение хи-квадрат для " << num_categs << " степеней свобод: " << criticalImportance;
			cout << "\n > Значение критерия хи-квадрат с " << num_categs << " степенью свободы: " << XiSquareDistribution << "\n";
			if (0 < XiSquareDistribution && XiSquareDistribution < criticalImportance) {
				cout << "\n[+++] Критерий пройден\n";
				return true;
			}
			else {
				cout << "\n[---] Критерий не пройден\n";
				return false;
			}
			
		}
		
		vector <pair <int, double>> ConflictCriterionPercentPoints(int m, int n) {
			vector <double> auxiliaryTableT = { 0.01, 0.05, 0.25, 0.50, 0.75, 0.95, 0.99, 1.0 };
			vector <double> auxiliaryA(n + 1, 0.0);
			vector <pair <int, double>> conflictsResP;
			auxiliaryA[1] = 1.0;
			int j0 = 1;
			int j1 = 1;
			for (int i = 0; i < n - 1; i++) {
				j1++;
				for (int j = j1; j > j0 - 1; j--) {
					double jm = j / (m * 1.0);
					auxiliaryA[j] = jm * auxiliaryA[j] + (1.0 + 1.0 / m - jm) * auxiliaryA[j - 1];
					if (auxiliaryA[j] < 1e-20) {
						auxiliaryA[j] = 0.0;
						if (j == j1) {
							j1--;
							continue;
						}
						if (j == j0)
						j0++;
					}
				}
			}
			int t = 0;
			int j = j0 - 1;
			double p = 0.0;
			while (t != auxiliaryTableT.size() - 1) {
				while (p <= auxiliaryTableT[t]) {
					j++;
					p += auxiliaryA[j];
				}
				conflictsResP.push_back(make_pair(n - j - 1, round((1 - p) * 1000) / 1000));
				t++;
			}
			reverse(conflictsResP.begin(), conflictsResP.end());
			return conflictsResP;
			
		}
		
		bool checkConflictCriterion(vector <double> allElems) {
			srand((unsigned int)time(0));
			double eps = 1e-20;
			vector <vector <pair<int, double>>> appropriatePercentPoints;
			vector <vector <int>> appropriatePointsTable;
			for (int bigSize = 8; bigSize < 21; bigSize++) {
				int nParameter = allElems.size() / bigSize;
				for (int dParameter = 2; dParameter < 9; dParameter++) {
					int j = 0;
					vector <int> normSeq;
					while (j < allElems.size()) {
						if (allElems[j] == 1.0)
						allElems[j] = 0.965;
						normSeq.push_back((floor(allElems[j] * dParameter)));
						j++;
					}
					set <vector <int>> words;
					int nDimension = 0;
					for (int jI = 0; jI < nParameter; jI++) {
						vector <int> dopSliced;
						for (int i = jI * bigSize; i < (jI + 1) * bigSize; i++) {
							dopSliced.push_back(normSeq[i]);
						}
						auto search = words.find(dopSliced);
						if (search == words.end())
						words.insert(dopSliced);
						else
						nDimension += 1;
					}
					for (int mParameter = 16; mParameter < 129; mParameter++) {
						int  mV = nParameter * mParameter;
						vector <pair <int, double>> confs_et_probs = ConflictCriterionPercentPoints(mV, nParameter);
						if (nDimension == 0 || confs_et_probs[0].first == -1 || confs_et_probs[0].first == 0)
						continue;
						if (confs_et_probs[2].first <= nDimension && nDimension <= confs_et_probs[confs_et_probs.size() - 2].first) {
							appropriatePercentPoints.push_back(confs_et_probs);
							vector <int> prom = { nDimension, bigSize, nParameter, dParameter, mParameter, mV };
							appropriatePointsTable.push_back(prom);
						}
					}
				}
			}
			int appropriatePercentPointsCount = appropriatePercentPoints.size();
			set <int> setRand;
			for (int j = 0; j < 5; j++) {
				
				int randInt = (double)(rand()) / RAND_MAX * appropriatePercentPointsCount - 1;
				auto search1 = setRand.find(randInt);
				if (search1 == setRand.end())
				setRand.insert(randInt);
				else {
					while (search1 == setRand.end()) {
						randInt = (double)(rand()) / RAND_MAX * appropriatePercentPointsCount - 1;
						search1 = setRand.find(randInt);
					}
				}
			}
			cout << "\n > Число параметров, при которых представленная последовательность удовлетворяет критерию конфликтов: " << appropriatePercentPointsCount;
			cout << "\n > Приведем случай, которая параметры подходят:";
			cout << "\n     Размерность вектора Vj: " << appropriatePointsTable[0][1];
			cout << "\n     Количество векторов: " << appropriatePointsTable[0][2];
			cout << "\n     Значение m: " << appropriatePointsTable[0][5];
			cout << "\n     Множитель для m: " << appropriatePointsTable[0][4];
			cout << "\n     Параметр нормирования d: " << appropriatePointsTable[0][3];
			cout << "\n     Количество возникших конфликтов: " << appropriatePointsTable[0][0];
			cout << "\n     Таблица процентных точек: ";
			for (int k = 0; k < appropriatePercentPoints[0].size(); k++) {
				cout << "(" << appropriatePercentPoints[0][k].first << ", " << appropriatePercentPoints[0][k].second << ") ";
			}
			cout << "\n";
			if (appropriatePercentPointsCount > 0) {
				cout << "\n[+++] Критерий пройден\n";
				return true;
			}
			else {
				cout << "\n[---] Критерий не пройден\n";
				return false;
			}
		}
		
		vector <bool> checkCriterions(string inputFileName, bool& errorMessage) {
			vector <bool> criteriaResults(7);
			ifstream inFile(inputFileName);
			string lineFile;
			vector <double> allElems;
			if (!inFile.is_open()) {
				cout << "Ошибка открытия файла " << inputFileName << endl;
				errorMessage = true;
				return criteriaResults;
			}
			string getlineElem;
			getline(inFile, getlineElem);
			stringstream ss(getlineElem);
			while (getline(ss, getlineElem, ',')) {
				if (getlineElem.size() > 1)
				getlineElem[1] = ',';
				double intElem = stod(getlineElem);
				allElems.push_back(intElem);
			}
			inFile.close();
			
			double expValue = calculateExpValue(allElems);
			cout << "Мат. ожидание: " << round(expValue * 1000) / 1000 << "\n";
			double standartDeviation = calculateStandartDeviation(allElems, expValue);
			cout << "Среднекв. отклонение: " << round(standartDeviation * 1000) / 1000 << "\n";
			double expValueError = calculateExpValueError(expValue);
			cout << "Погрешность мат. ожидания: " << round(expValueError * 1000) / 1000 << "\n";
			double standartDeviationError = calculateStandartDeviationError(standartDeviation);
			cout << "Погрешность среднекв. отклонения: " << round(standartDeviationError * 1000) / 1000 << "\n";
			
			bool res = false;
			cout << "\n ----------------------------------------------------------------------- \n";
			cout << "\n Критерий Хи-квадрат \n\n";
			criteriaResults[0] = checkXiSquareCriterion(allElems);
			cout << "\n ----------------------------------------------------------------------- \n";
			cout << "\n Критерий серий \n\n";
			criteriaResults[1] = checkSeriesCriterion(allElems);
			cout << "\n ----------------------------------------------------------------------- \n";
			cout << "\n Критерий интервалов \n\n";
			criteriaResults[2] = checkIntervalsCriterion(allElems);
			cout << "\n ----------------------------------------------------------------------- \n";
			cout << "\n Критерий разбиений \n\n";
			criteriaResults[3] = checkPartitionCriterion(allElems);
			cout << "\n ----------------------------------------------------------------------- \n";
			cout << "\n Критерий перестановок \n\n";
			criteriaResults[4] = checkPermutationsCriterion(allElems);
			cout << "\n ----------------------------------------------------------------------- \n";
			cout << "\n Критерий монотонности \n\n";
			criteriaResults[5] = checkMonotonicityCriterion(allElems);
			cout << "\n ----------------------------------------------------------------------- \n";
			cout << "\n Критерий конфликтов \n\n";
			criteriaResults[6] = checkConflictCriterion(allElems);
			cout << "\n ----------------------------------------------------------------------- \n";
			
			return criteriaResults;
		}
		
		void coutCriterionsResult(vector <bool> criteriaResults) {
			vector <string> criteriaNames = { "Критерий хи-квадрат", "Критерий серий",  "Критерий интервалов", "Критерий разбиений",  "Критерий перестановок",  "Критерий монотонности",  "Критерий конфликтов"};
			cout << "\nРезультат тестирования статистических свойств последовательности ПСЧ: \n\n";
			for (int i = 0; i < criteriaResults.size(); i++)
			{
				cout << criteriaNames[i] << ": ";
				if (criteriaResults[i])
				cout << "+++" << endl;
				else
				cout << "---" << endl;
			}
			return;
		}
		
		int main(int argc, char* argv[])
		{
			/*
			Программа для тестирования статистических свойств последовательности ПСЧ
			*/
			setlocale(LC_ALL, "Rus");
			string inputFileName = "rntInput.dat";
			for (int i = 0; argv[i]; i++)
			{
				string checkStr = string(argv[i]);
				if (findInStr(checkStr, 2) == "/h") {
					helpFunc();
					return 0;
				}
				if (checkStr.length() > 2) {
					string ifStr = findInStr(checkStr, 3);
					string subStr = checkStr.substr(3, checkStr.length());
					if (ifStr == "/f:") {
						inputFileName = subStr;
					}
				}
			}
			bool errorMessage = false;
			vector <bool> resString = checkCriterions(inputFileName, errorMessage);
			if (errorMessage)
			return 0;
			
			cout << "Input File: " << inputFileName << endl;
			
			coutCriterionsResult(resString);
			return 0;
		}
	\end{minted}
		
	\section{Исходный код программы построения графиков}
	
	\begin{minted}{python}
		import matplotlib.pyplot as plt
		import math
		
		def expectedValue(vec):
			val = 0
			for i in vec:
				val = val + i
			val = val / len(vec)
			return val
		
		f = open("D:\\cpp_projects\\rnt\\Debug\\stOutputLc.dat", "r")
		#f = open("D:\\cpp_projects\\rnt\\Debug\\stOutputAdd.dat", "r")
		#f = open("D:\\cpp_projects\\rnt\\Debug\\stOutput5p.dat", "r")
		#f = open("D:\\cpp_projects\\rnt\\Debug\\stOutputLfsr.dat", "r")
		#f = open("D:\\cpp_projects\\rnt\\Debug\\stOutputNfsr.dat", "r")
		#f = open("D:\\cpp_projects\\rnt\\Debug\\stOutputMt.dat", "r")
		#f = open("D:\\cpp_projects\\rnt\\Debug\\stOutputRc4.dat", "r")
		#f = open("D:\\cpp_projects\\rnt\\Debug\\stOutputRsa.dat", "r")
		#f = open("D:\\cpp_projects\\rnt\\Debug\\stOutputBbs.dat", "r")
		currElem = []
		mVec = []
		dVec = []
		nVec = []
		acc = 0
		allElems = list(map(float, (f.read()).split(',')))
		for a in allElems:
			currElem.append(float(a))
			if (len(currElem) % 50 == 0):
				m = expectedValue(currElem)
				dev = 0
				for i in currElem:
					dev = dev + ((i - m) ** 2)
				dev = dev / (len(currElem) - 1)
				d = math.sqrt(dev)
				acc = acc + 50
				mVec.append(round(m, 3))
				dVec.append(round(d, 3))
				nVec.append(acc)
		fig, (ax1, ax2) = plt.subplots(2, 1, figsize=(10,10))
		for i in range(len(nVec)):
			ax1.scatter(nVec[i], mVec[i], c = 'b', s = 1)
			ax2.scatter(nVec[i], dVec[i], c = 'b', s = 1)
		ax1.set_title("Математическое ожидание")
		ax2.set_title("Среднеквадратическое отклонение")
		plt.show()
	\end{minted}
	%\includepdf[pages=-]{pril-A.pdf}

	%\includepdf[pages=-]{pril-B.pdf}
	
\end{document}